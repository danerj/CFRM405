\documentclass[11pt]{article}
\usepackage{amsmath}
\usepackage{amssymb}

\begin{document}
\title{CFRM Homework 4}
\author{Dane Johnson}
\date{\today}
\maketitle

1. Let $A = \begin{bmatrix}
1 & 1 & 0 \\ 1 & 4 & 1 \\ -2 & 1 & 1
\end{bmatrix} \,. $ \\\\

(a) $ A
\sim \begin{bmatrix}
1 & 1 & 0 \\ 0 & 3 & 1 \\ 0 & 3 & 1
\end{bmatrix}
\sim \begin{bmatrix}
1 & 1 & 0 \\ 0 & 3 & 1 \\ 0 & 0 & 0
\end{bmatrix} \,.\; A
$ is now upper triangular with two pivots. \\

(b) Let $ b = (1,6,3)^T $ (to match dimensions). The system $A\vec{x} = \vec{b}$ can be solved by augmenting $A$ with $b$:

$$
[A|\vec{b}] = \begin{bmatrix}
1 & 1 & 0 & 1 \\ 1 & 4 & 1 & 6 \\ -2 & 1 & 1 & 3
\end{bmatrix}
\sim \begin{bmatrix}
1 & 1 & 0 & 1 \\ 0 & 3 & 1 & 5 \\ 0 & 3 & 1 & 5
\end{bmatrix}
\sim \begin{bmatrix}
1 & 1 & 0 & 1 \\ 0 & 3 & 1 & 5 \\ 0&0&0&0
\end{bmatrix} \;.
$$
This shows that the system has infinite solutions. If we let $\vec{x} = (x_1, x_2, x_3)^T$, then from the reduce matrix above we know that the solutions have the form $x_2 = \frac{5-x_3}{3}, x_1 = 1- x_2 = 1 - \frac{5-x_3}{3}$ where $x_3 \in \mathbb{R}$ is a free variable (although everything here is real we could actually even say $x_3 \in \mathbb{C}$).\\

(c) Let $ b = (1,6,5)^T $ (to match dimensions). The system $A\vec{x} = \vec{b}$ can be solved by augmenting $A$ with $b$:

$$
[A|\vec{b}] = \begin{bmatrix}
1 & 1 & 0 & 1 \\ 1 & 4 & 1 & 6 \\ -2 & 1 & 1 & 5
\end{bmatrix}
\sim \begin{bmatrix}
1 & 1 & 0 & 1 \\ 0 & 3 & 1 & 5 \\ 0 & 3 & 1 & 7
\end{bmatrix}
\sim \begin{bmatrix}
1 & 1 & 0 & 1 \\ 0 & 3 & 1 & 5 \\ 0&0&0&2
\end{bmatrix} \;.
$$
This system has no solutions. The reason is that we have found through reduction that in order for $\vec{x} = (x_1,x_2,x_3)^T$ to be a solution,  it must be the case that $0x_3 = 0 = 2$ according to the reduced matrix's last row. But $\nexists \, x_3$ such that $0x_3 = 2$. So the system is inconsistent and has no solution. \\

(d) Since the system described in part (b) has infinite solutions, all of the form $x_2 = \frac{5-x_3}{3}, x_1 = 1 - \frac{5-x_3}{3}$ for $\vec{x} = (x_1, x_2, x_3)^T$,
we get two solutions by taking first $x_3 = 5$ and then $x_3 = 0$:
$$ \begin{pmatrix}
1 \\ 0 \\ 5
\end{pmatrix} \;,
\begin{pmatrix}
\frac{2}{3} \\[2pt] \frac{5}{3} \\[2pt] 0
\end{pmatrix} \;.$$

However, as we have seen, the system described in part (c) has no solutions to find.\\

(e) $A$ is not invertible. There are many equivalent ways of seeing this, but in this problem we have established that it is not true that for each $\vec{b} \in \mathbb{R}^3$ that there exists a unique solution $\vec{x}$ to the equation $A\vec{x} = \vec{b}$. We have seen that there may actually be no solutions or infinite solutions. So $A$, which executes a map $f : \mathbb{R}^3 \longrightarrow \mathbb{R}^3$ (or $\mathbb{C}^3$ if we like) that is not a bijection, is not an invertible matrix.\\

2. Suppose $AB = I, CA = I$, where $I$ is the $nxn$ identity matrix.\\

(a) $A,B,C$ are all $nxn$ matrices. To see this note that $AB = I$ shows us that $A$ is $nxr$ and $B$ is $rxn$ (where $r$ is yet an unknown postive integer). Also, $CA = I$ shows us that $C$ is of dimension $nxq$ and $A$ is of dimension $qxn$. But then since we know $q=n$ from the last sentence we may conclude $A$ and $C$ must both be $nxn$. Finally since we now know that $A$ is $nxn$ we must have $B$ of dimension $nxn$ so that $AB$ is defined.\\

(b) $B = IB = (CA)B = C(AB) = CI = C$. Therefore $B = C$. \\

(c) The matrix $A$ is invertible iff there exists a matrix, denoted $A^{-1}$ with the property that $AA^{-1} = A^{-1}A = I$. From parts (b), (c) we see that the matrix $B \; (=C)$ has this property. So $A$ is invertible and $A^{-1} = B$.\\

3. Let $A$ be a square matrix such that $A^2 = A$ (i.e. $A$ is a projector). Then we have the following equalites:

$$ (I - A)^2 = I^2 - 2IA + A^2 = I - 2A + A = I - A$$
\begin{align*}
(I-A)^7 & = (I-A)(I-A)^2(I-A)^2(I-A)^2 \\ 
&= (I-A)(I-A)(I-A)(I-A) \\
&= (I-A)^2(I-A)^2 \\ 
&= (I-A)(I-A) \quad (by\,the\,last\,part)\\ 
&= (I-A)^2 \\
&= (I-A)
\end{align*}

4.\\

(a)
$$ -3\begin{pmatrix}
1 \\ 2 \\ 3
\end{pmatrix}
+ 2\begin{pmatrix}
6 \\ 4 \\ 2
\end{pmatrix}
= \begin{pmatrix}
9 \\ 2 \\ -5
\end{pmatrix}
$$

(b)
$$ A = \begin{bmatrix}
1 & 6 & 9 \\ 2 & 4 & 2 \\ 3 & 2 & -5
\end{bmatrix}
\sim \begin{bmatrix}
1 & 0 & -3 \\ 0 & 1 & 2 \\ 0 & 0 & 0
\end{bmatrix}$$
From this reduction (which I did on paper to solve part (a)) we see that there are two pivots.\\

5. Suppose $A$ is a 6x20 matrix and $B$ is a 20x7 matrix.\\

(a) The dimensions of $C = AB$ are 6x7.\\

(b) Suppose $A$,$B$, and $C$ are partitioned as shown below and that we know $A_{11}$ is 2x10, $B_{22}$ is 4x3 and $C$ is ?x4.

\[
  A = \left[\begin{array}{@{}c|c|c@{}}
    A_{11} & A_{12} & A_{13} \\ \hline
    A_{21} & A_{22} & A_{23}
  \end{array}\right]
\]
\[
  B = \left[\begin{array}{@{}c|c@{}}
    B_{11} & B_{12} \\ \hline
    B_{21} & B_{22}  \\ \hline
    B_{31} & B_{32}
  \end{array}\right]
\]
\[
  C = \left[\begin{array}{@{}c|c@{}}
    C_{11} & C_{12} \\ \hline
    C_{21} & C_{22}  \\
  \end{array}\right]
\]

Since $A_{11}$ is 2x10 and $C$ is ?x4, $B_{11}$ must be 10x4. So $C_{11}$ is 2x4. Since $B_{22}$ is 4x3, $B_{21}$ must be 4x4 and so $A_{12}$ is 2x4. We can continue to fill in missing information making the assumption that if, for example, $A_{11}$ has $2$ rows then so must $A_{12}$ and $A_{13}$. Similarly, if $B_{11}$ has $4$ columns then so must both the blocks beneath it. This leads to the following dimensions:
\[
  A = \left[\begin{array}{@{}c|c|c@{}}
    $2x10$ & $2x4$ & $2x6$ \\ \hline
    $4x10$ & $4x4$ & $4x6$
  \end{array}\right]
\]
\[
  B = \left[\begin{array}{@{}c|c@{}}
    $10x4$ & $10x3$ \\ \hline
    $4x4$ & $4x3$  \\ \hline
    $6x4$ & $6x3$
  \end{array}\right]
\]
\[
  C = \left[\begin{array}{@{}c|c@{}}
    $2x4$ & $2x3$ \\ \hline
    $4x4$ & $4x3$  \\
  \end{array}\right]
\]

where it should be clear that each block in the partition scheme from before has the dimensions corresponding to its location in this second set.\\

(c) Writing out the blocks of $C$ as the result of the block multiplications from the last part we have:

\[
  C = \left[\begin{array}{@{}c|c@{}}
    A_{11}B_{11}+A_{12}B_{21}+A_{13}B_{31} & A_{11}B_{12}+A_{12}B_{22}+A_{13}B_{32} \\ \hline
    A_{21}B_{11}+A_{22}B_{21}+A_{23}B_{31} & A_{21}B_{12}+A_{22}B_{22}+A_{23}B_{32} \\
  \end{array}\right] \,.
\]

6. Let $A$ be an $m \times n$ matrix.\\

(a) If $\widetilde{Q}\widetilde{R} = A$, the smallest possible $\widetilde{Q}$ is $m \times
m$ and the smallest possible $\widetilde{R}$ is $m \times n$. Then $\widetilde{Q}\widetilde{R}$ still results in an $m \times n$ matrix. The reason that $\widetilde{Q}$ is $m \times m$ is because its number or rows must equal the number of rows of $A$ and since $\widetilde{Q}$ is orthogonal (or perhaps unitary if we are dealing with complex matrices)it must be square. Because $\widetilde{Q}$ is $m \times m$ this forces $\widetilde{R}$ to have $m$ rows and to also $\widetilde{R}$ must have the same number of columns as $A$.\\

(b)
\begin{align*}
d_i^2 &= \tilde{x}_i^T \hat{S}^{-1} \tilde{x}_i \\
	  &= \tilde{x}_i^T \left(\frac{1}{m-1}\tilde{A}^T \tilde{A} \right)^{-1} \tilde{x}_i \\
	  &= \tilde{x}_i^T \left(\frac{1}{m-1}(\tilde{Q}\tilde{R})^T\tilde{Q}\tilde{R}\right)^{-1} \tilde{x}_i \\
	  &= \tilde{x}_i^T \left(\frac{1}{m-1} \tilde{R}^T\tilde{Q}^T\tilde{Q}\tilde{R}\right)^{-1} \tilde{x}_i \\
	  &= \tilde{x}_i^T \left(\frac{1}{m-1} \tilde{R}^TI\tilde{R}\right)^{-1} \tilde{x}_i \\
	  &= \tilde{x}_i^T \left(\frac{1}{m-1} \tilde{R}^T\tilde{R}\right)^{-1} \tilde{x}_i \\
	  &= \tilde{x}_i^T (m-1) \left(\tilde{R}^T\tilde{R}\right)^{-1} \tilde{x}_i \,. \\
\end{align*}

Since the instructions say not to invert any matrices this is as far as I can go.

\end{document}